\documentclass[12pt,letterpaper,titlepage]{report}

% packages

\usepackage[utf8]{inputenc}
\usepackage{mathptmx}
\usepackage{geometry}
\usepackage{float}
\usepackage{subcaption}
\usepackage{multirow}
\usepackage{makecell}

% commands

\newcommand{\myTitle}{Series and Parallel Circuits}
\newcommand{\myName}{Shawn Lutch}
\newcommand{\myPeriod}{PHYS 2240.502}

% metadata

\usepackage[pdftex,
            pdfauthor={\myName{}},
            pdftitle={\myTitle{}},
            pdfsubject={\myPeriod{}},
            pdfproducer={\myName{} via LaTeX},
            pdfcreator={pdflatex}]{hyperref}
			
% layout

\pagenumbering{gobble}
\raggedright
\geometry{
	letterpaper,
	lmargin=0.75in,
	rmargin=0.75in,
	tmargin=1.0in,
	bmargin=1.0in
}

% document

\begin{document}


%% title page


\title{\myTitle{}}
\author{\myName{}\\ \myPeriod{}}
\date{\today}
\maketitle


%% abstract


\section*{Abstract}


%% introduction

\bigskip
\section*{Introduction}


%% apparatus

\bigskip
\section*{Apparatus}

\begin{itemize}
	\item PASCO Capstone (data acquisition, display, analysis software)
	\item 850 Universal Interface
	\item AC/DC Electronics Laboratory
	\item Patch cords (x8)
	\item Resistors (x6)
		\begin{itemize}
			\item 100 $\Omega$ (brown-black-brown-gold) resistors (x2)
			\item 330 $\Omega$ (orange-orange-brown-gold) resistors (x2)
			\item 560 $\Omega$ (green-blue-brown-gold) resistors (x2)
		\end{itemize}
\end{itemize}


%% procedure

\bigskip
\section*{Procedure}

Given that the resistors in the circuit diagrams are labeled $R_{1\ldots6}$, we first separated and labeled the resistors as instructed by the lab manual:

\bigskip
\begin{minipage}{\linewidth}
\centering
\begin{tabular}{ | c | c | c | c | c | c | }
	\hline
	$R_1$ & $R_2$ & $R_3$ & $R_4$ & $R_5$ & $R_6$ \\
	\hline
	330 $\Omega$ & 560 $\Omega$ & 100 $\Omega$ & 100 $\Omega$ & 560 $\Omega$ & 330 $\Omega$ \\
	\hline
\end{tabular}
\end{minipage}
\bigskip

The precision of each resistor is $\pm 5 \%$, as indicated by the gold band on each. In order to ensure the highest level of accuracy possible in our calculations for the equivalent resistance ($R_{eq}$) of each circuit, we first needed to find the actual resistance of our six resistors, as shown in Table 1. This method improves the precision of our values for $R_{1\ldots6}$ to around $\pm 1 \%$. These numbers would be used later in our calculations for the theoretical $R_{eq}$ of each circuit.

%% data

\bigskip
\section*{Data}


% resistor calibration
\begin{minipage}{\linewidth}
\centering
\captionof{table}{Measured Resistance Values}
\begin{tabular}{ | c | c | c | c | c | c | } \hline
	$R_1 = 330 \Omega$ & $R_2 = 560 \Omega$ & $R_3 = 100 \Omega$ & $R_4 = 100 \Omega$ & $R_5 = 560 \Omega$ & $R_6 = 330 \Omega$ \\ \hline
	$315 \Omega$ & $535 \Omega$ & $95.7 \Omega$ & $96.8 \Omega$ & $537 \Omega$ & $314 \Omega$ \\ \hline
\end{tabular}
\end{minipage}

\bigskip
\bigskip

% ammeter calibration
\begin{minipage}{\linewidth}
\centering
\captionof{table}{Ammeter Calibration Data}
\begin{tabular}{ | c | c | c | c | } \hline

	\thead{Voltage \\ (V)} & \thead{Ideal $I$ \\ (mA)} & \thead{Measured $I$ \\ (mA)} & \thead{Correction \\ (mA)} \\ \hline
	
	0 & 0    & 0.3  & -0.3  \\ \hline
	1 & 10.3 & 10.6 & -0.3  \\ \hline
	2 & 20.7 & 20.9 & -0.2  \\ \hline
	3 & 31.0 & 31.0 & 0.0   \\ \hline
	4 & 41.3 & 41.3 & 0.0   \\ \hline
	5 & 51.7 & 51.6 & 0.1   \\ \hline
	6 & 62.0 & 62.1 & -0.1  \\ \hline
	7 & 72.3 & 72.5 & -0.2  \\ \hline

\end{tabular}
\end{minipage}

\bigskip
\bigskip

\begin{minipage}{\linewidth}
\centering
\captionof{table}{Resistance Summary}
\begin{tabular}{ | c | c | c | c | } \hline
     \thead{Circuit} & \thead{Theoretical \\ $R_{eq}$ ($\Omega$)} & \thead{Measured \\ Current (A)} & \thead{Corrected \\ Current (A)} \\ \hline
\end{tabular}
\end{minipage}


%% calcs and graphs

\bigskip
\section*{Calculations and Graphs}


%% discussion

\bigskip
\section*{Discussion of Results and Error Analysis}


%% conclusion

\bigskip
\section*{Conclusion}


%% done...


\end{document}