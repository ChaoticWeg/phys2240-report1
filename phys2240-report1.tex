\documentclass[12pt,letterpaper,titlepage]{report}

% packages

\usepackage[utf8]{inputenc}
\usepackage{mathptmx}
\usepackage{geometry}
\usepackage{float}
\usepackage{subcaption}
\usepackage{multirow}
\usepackage{makecell}
\usepackage{circuitikz}

% commands

\newcommand{\myTitle}{Series and Parallel Circuits}
\newcommand{\myName}{Shawn Lutch}
\newcommand{\myPeriod}{PHYS 2240.502}

% shamelessly copied and pasted from https://tex.stackexchange.com/a/229158

\ctikzset{bipoles/ammeter/text rotate/.initial=0,
rotation/.style={bipoles/ammeter/text rotate=#1}}

% code from pgfcircbipoles.sty

\makeatletter
\pgfcircdeclarebipole{}{\ctikzvalof{bipoles/ammeter/height}}{ammeter}{\ctikzvalof{bipoles/ammeter/height}}{\ctikzvalof{bipoles/ammeter/width}}{
    \def\pgf@circ@temp{right}
    \ifx\tikz@res@label@pos\pgf@circ@temp
        \pgf@circ@res@step=-1.2\pgf@circ@res@up
    \else
        \def\pgf@circ@temp{below}
        \ifx\tikz@res@label@pos\pgf@circ@temp
            \pgf@circ@res@step=-1.2\pgf@circ@res@up
        \else
            \pgf@circ@res@step=1.2\pgf@circ@res@up
        \fi
    \fi

    \pgfpathmoveto{\pgfpoint{\pgf@circ@res@left}{\pgf@circ@res@zero}}       
    \pgfpointorigin \pgf@circ@res@other =  \pgf@x  \advance \pgf@circ@res@other by -\pgf@circ@res@up
    \pgfpathlineto{\pgfpoint{\pgf@circ@res@other}{\pgf@circ@res@zero}}
    \pgfusepath{draw}

    \pgfsetlinewidth{\pgfkeysvalueof{/tikz/circuitikz/bipoles/thickness}\pgfstartlinewidth}

        \pgfscope
            \pgfpathcircle{\pgfpointorigin}{.9\pgf@circ@res@up}
            \pgfusepath{draw}       
        \endpgfscope    

    \pgftransformrotate{\ctikzvalof{bipoles/ammeter/text rotate}}% <= magic line
    \pgfsetlinewidth{\pgfstartlinewidth}

    \pgfsetarrowsend{latex}
    \pgfpathmoveto{\pgfpoint{\pgf@circ@res@other}{\pgf@circ@res@down}}
    \pgfpathlineto{\pgfpoint{-\pgf@circ@res@other}{\pgf@circ@res@up}}
    \pgfusepath{draw}
    \pgfsetarrowsend{}


    \pgfpathmoveto{\pgfpoint{-\pgf@circ@res@other}{\pgf@circ@res@zero}}
    \pgfpathlineto{\pgfpoint{\pgf@circ@res@right}{\pgf@circ@res@zero}}
    \pgfusepath{draw}


    \pgfnode{circle}{center}{\textbf{A}}{}{}
}
\makeatother % allow rotation of elements in circuit diagram

% metadata

\usepackage[pdftex,
            pdfauthor={\myName{}},
            pdftitle={\myTitle{}},
            pdfsubject={\myPeriod{}},
            pdfproducer={\myName{} via LaTeX},
            pdfcreator={pdflatex}]{hyperref}
			
% layout

\pagenumbering{gobble}
\raggedright
\geometry{
	letterpaper,
	lmargin=0.75in,
	rmargin=0.75in,
	tmargin=1.0in,
	bmargin=1.0in
}

% document

\begin{document}


%% title page


\title{\myTitle{}}
\author{\myName{}\\ \myPeriod{}}
\date{\today}
\maketitle


%% abstract


\section*{Abstract}


%% introduction

\bigskip
\section*{Introduction}


%% apparatus

\bigskip
\section*{Apparatus}

\begin{itemize}
	\item PASCO Capstone (data acquisition, display, analysis software)
	\item 850 Universal Interface
	\item AC/DC Electronics Laboratory
	\item Patch cords (x8)
	\item Resistors (x6)
		\begin{itemize}
			\item 100 $\Omega$ (brown-black-brown-gold) resistors (x2)
			\item 330 $\Omega$ (orange-orange-brown-gold) resistors (x2)
			\item 560 $\Omega$ (green-blue-brown-gold) resistors (x2)
		\end{itemize}
\end{itemize}


%% procedure

\bigskip
\section*{Procedure}

Given that the resistors in the circuit diagrams are labeled $R_{1\ldots6}$, we first separated and labeled the resistors as instructed by the lab manual:

\bigskip
\begin{minipage}{\linewidth}
\centering
\begin{tabular}{ | c | c | c | c | c | c | }
	\hline
	$R_1$ & $R_2$ & $R_3$ & $R_4$ & $R_5$ & $R_6$ \\
	\hline
	330 $\Omega$ & 560 $\Omega$ & 100 $\Omega$ & 100 $\Omega$ & 560 $\Omega$ & 330 $\Omega$ \\
	\hline
\end{tabular}
\end{minipage}
\bigskip

The precision of each resistor is $\pm 5 \%$, as indicated by the gold band on each. In order to ensure the highest level of accuracy possible in our calculations for the equivalent resistance ($R_{eq}$) of each circuit, we first needed to find the actual resistance of our six resistors, as shown in Table 1. These numbers would be used later in our calculations for the theoretical $R_{eq}$ of each circuit. The following circuit was used to calibrate each resistor $R_n$ by using PASCO Capstone to record the actual resistance to a precision of around $\pm 1 \%$:

\bigskip
\begin{minipage}{\linewidth}
\centering
\begin{circuitikz}
\draw
(2,2) to[battery,l=850 Interface] (2,0)
      to[ammeter,rotation=180] (0,0)
      to[resistor,l=$R_n$] (0,2) -- (2,2)
;
\end{circuitikz}
\end{minipage}
\bigskip



%% data

\bigskip
\section*{Data}


% resistor calibration
\begin{minipage}{\linewidth}
\centering
\captionof{table}{Measured Resistance Values}
\begin{tabular}{ | c | c | c | c | c | c | } \hline

    \thead{$R_n$} & \thead{Ideal \\ $R$ ($\Omega$)} & \thead{Measured \\ $R$ ($\Omega$)} \\ \hline
    $R_1$ & 330 & 315  \\ \hline
    $R_2$ & 560 & 535  \\ \hline
    $R_3$ & 100 & 95.7 \\ \hline
    $R_4$ & 100 & 96.8 \\ \hline
    $R_5$ & 560 & 537  \\ \hline
    $R_6$ & 330 & 314  \\ \hline
    
\end{tabular}
\end{minipage}

\bigskip
\bigskip

% ammeter calibration
\begin{minipage}{\linewidth}
\centering
\captionof{table}{Ammeter Calibration Data}
\begin{tabular}{ | c | c | c | c | } \hline
	\thead{Voltage (V)} & \thead{Theoretical \\ Current (mA)} & \thead{Measured \\ Current (mA)} & \thead{Correction \\ (mA)} \\ \hline
	0 V & 0 mA & 0.3 mA & -0.3 mA \\ \hline
	1 V & 10.3 mA & 10.6 mA & -0.3 mA \\ \hline
	2 V & 20.7 mA & 20.9 mA & 0.2 mA \\ \hline
	3 V & 31.0 mA & 31.0 mA & 0.0 mA \\ \hline
	4 V & 41.3 mA & 41.3 mA & 0.0 mA \\ \hline
	5 V & 51.7 mA & 51.6 mA & 0.1 mA \\ \hline
	6 V & 62.0 mA & 62.1 mA & -0.1 mA \\ \hline
	7 V & 72.3 mA & 72.5 mA & -0.2 mA \\ \hline
\end{tabular}
\end{minipage}

\bigskip
\bigskip

\begin{minipage}{\linewidth}
\centering
\captionof{table}{Resistance Summary}
\begin{tabular}{ | c | c | c | c | } \hline
     \thead{Circuit} & \thead{Theoretical \\ $R_{eq}$ ($\Omega$)} & \thead{Measured \\ Current (A)} & \thead{Corrected \\ Current (A)} \\ \hline
\end{tabular}
\end{minipage}


%% calcs and graphs

\bigskip
\section*{Calculations and Graphs}


%% discussion

\bigskip
\section*{Discussion of Results and Error Analysis}


%% conclusion

\bigskip
\section*{Conclusion}


%% done...


\end{document}